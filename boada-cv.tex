\documentclass[margin,line, 11pt]{res}

\topmargin=-0.5in    % Make letterhead start about 1 inch from top of page
\textheight=10in  % text height can be bigger for a longer letter
\oddsidemargin -.5in
%\evensidemargin -.5in
\textwidth=6.0in
\itemsep=0in
\parsep=0in

\nofiles

% if using pdflatex:
\setlength{\pdfpagewidth}{\paperwidth}
\setlength{\pdfpageheight}{\paperheight}

\newenvironment{list1}{
  \begin{list}{\ding{113}}{%
      \setlength{\itemsep}{0in}
      \setlength{\parsep}{0in} \setlength{\parskip}{0in}
      \setlength{\topsep}{0in} \setlength{\partopsep}{0in}
      \setlength{\leftmargin}{0.17in}}}{\end{list}}
\newenvironment{list2}{
  \begin{list}{$\bullet$}{%
      \setlength{\itemsep}{0in}
      \setlength{\parsep}{0in} \setlength{\parskip}{0in}
      \setlength{\topsep}{0in} \setlength{\partopsep}{0in}
      \setlength{\leftmargin}{0.2in}}}{\end{list}}

\renewcommand{\familydefault}{\sfdefault}
%\usepackage[sfdefault, light, condensed]{roboto}
\usepackage[light,condensed]{iwona}
\usepackage[T1]{fontenc}
\usepackage{hyperref}

\begin{document}

\name{{\huge\bf Steven Boada, Ph.D}\vspace*{.1in}}

\begin{resume}
%\vspace*{-3mm}
\section{Contact Information}
\begin{tabular}{@{}p{3in}p{3in}}
Cranford, New Jersey, USA & \href{mailto:stevenboada@gmail.com}{stevenboada@gmail.com} \\
(615) 200-0119   & \url{http://boada.github.io} \\
\end{tabular}
\vspace*{-6mm}

\section{Profile}
Collaborative, scientific thinker passionate about discovering and communicating nuanced insight from complicated data. Strong programming and analytical background working with large, heterogeneous, and often noisy datasets.
\vspace*{-5mm}

\section{Education}
\textbf{Texas A\&M University}, College Station, Texas USA\\
\vspace*{-5mm}
\begin{list1}
	\item[]Ph.D, Physics (Astronomy focus), August, 2016
	\begin{list2}
		\item Dissertation Title: ``Measuring the Scatter in the Cluster Optical Richness--Mass Relation with Machine Learning''
	\end{list2}
\end{list1}
\vspace*{-5mm}

\textbf{The University of Tennessee}, Knoxville, Tennessee USA\\
\vspace*{-5mm}
\begin{list1}
	\item[] M.S., Physics (Astronomy focus),  August, 2009
	\begin{list2}
		\item Thesis Title: ``An Automated Approach to the Study and Classification of Colliding and Interacting Galaxies''
	\end{list2}
	\item[] B.S., Physics,  May, 2007
\end{list1}
\vspace*{-5mm}

\section{Technical Skills}
\textbf{Machine Learning:} Regression (Linear, Random Forests), classification (RF, SVM), feature engineering, optimization, deep learning\\
\textbf{Statistical Methods:} Hypothesis testing and confidence intervals, error analysis, image analysis, Monte Carlo methods (e.g., \href{https://github.com/dfm/emcee}{emcee})\\
\textbf{Software and Computing:}  Python (e.g. Scikit-learn, Numpy, Scipy, Pandas, Matplotlib, fast.ai), SQL, ANSI C, Linux, Microsoft Excel, GPGPU, and HPC (100k+ core) applications\\
\vspace*{-10mm}

\section{Data Projects}
\textbf{Using Imaging to Predict Galaxy Spectroscopic Properties}\newline
    \begin{list2}
    	\vspace*{-5mm}
    	\item Leveraged Convolution Neural Networks, trained on GPUs, to analyze $\sim150,000$ images from the Sloan Digital Sky Survey. See \url{https://github.com/boada/galaxy-cnns}.
    	\item Predicted spectroscopic properties with $\sim5\%$ error from psuedo-three color imaging.
    \end{list2}
    \vspace*{-4mm}
\textbf{Predicting Tournament Performance in Warmachine}\newline
    \begin{list2}
    	\vspace*{-5mm}
    	\item Wrangled tournament results of a popular tabletop game using Python (e.g., Pandas).
    	\item Created an Elo based model to forecast the results of an upcoming tournament.
    	\item Explored tournament entries for insights into and to identify potential problematic future matches.
    \end{list2}
\vspace*{-4mm}

\section{Professional \newline Experience}
\textbf{Rutgers University}, New Brunswick, New Jersey USA \newline
\textit{Postdoctoral Research Associate} \hfill \textbf{September, 2016 - Present}\newline
    \begin{list2}
    	\vspace*{-5mm}
    	\item Designed and built a pipeline to analyzed TBs of astronomical imaging; producing calibrated, standardized data catalogs.
    	\item Led weekly collaboration meetings with senior scientists and supervised student research.
    	\item Contributed to open source projects including: \href{https://github.com/boada/photometrypipeline}{\sc{photometrypipeline}}, \href{http://astlib.sourceforge.net/}{\sc{astLib}}, and \href{https://github.com/boada/easyGalaxy}{\sc{easyGalaxy}}.
    \end{list2}
\vspace*{-4mm}

\textbf{Texas A\&M University}, College Station, Texas USA\newline
\textit{Ph.D Candidate} \hfill \textbf{August, 2010 - 2016}\newline
    \begin{list2}
    	\vspace*{-5mm}
    	\item Conducted original research of a forthcoming astronomical survey and showed that (simulated) results could be improved by implementing machine learning techniques (e.g., random forest regression) when compared to traditional analysis methods. Implemented these ML methods and produced improved results in a pilot survey of the real sky and under real-world conditions.
    	\item Collaborated with group members both in person, and through collaborative tools (e.g., GitHub, SVN).
    	\item Presented scientific results in high-impact, peer reviewed journals and at international conferences.
    \end{list2}
\vspace*{-4mm}

\textbf{The University of Tennessee}, Knoxville, Tennessee USA\newline
\textit{Master's Candidate} \hfill \textbf{August, 2007 - 2009}\newline
    \begin{list2}
    	\vspace*{-5mm}
    	\item Conducted original research at the National Center for Computational Science, part of Oak Ridge National Laboratory, using HPC ($\sim100$k cores) scientific simulations.
      \item Optimized simulation initial conditions using a genetic algorithm based search.
      \item Implemented a computer vision algorithm to automatically analyze hundreds of GBs of simulation results.
    \end{list2}
\vspace*{-4mm}

\end{resume}
\end{document}
