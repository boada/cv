\documentclass[margin,line, 11pt]{res}

\topmargin=-0.5in    % Make letterhead start about 1 inch from top of page
\textheight=10in  % text height can be bigger for a longer letter
\oddsidemargin -.5in
\evensidemargin -.5in
\textwidth=6.0in
\itemsep=0in
\parsep=0in

\nofiles

% if using pdflatex:
\setlength{\pdfpagewidth}{\paperwidth}
\setlength{\pdfpageheight}{\paperheight}

\newenvironment{list1}{
  \begin{list}{\ding{113}}{%
      \setlength{\itemsep}{0in}
      \setlength{\parsep}{0in} \setlength{\parskip}{0in}
      \setlength{\topsep}{0in} \setlength{\partopsep}{0in}
      \setlength{\leftmargin}{0.17in}}}{\end{list}}
\newenvironment{list2}{
  \begin{list}{$\bullet$}{%
      \setlength{\itemsep}{0in}
      \setlength{\parsep}{0in} \setlength{\parskip}{0in}
      \setlength{\topsep}{0in} \setlength{\partopsep}{0in}
      \setlength{\leftmargin}{0.2in}}}{\end{list}}

\renewcommand{\familydefault}{\sfdefault}
%\usepackage[sfdefault, light, condensed]{roboto}
\usepackage[light,condensed]{iwona}
\usepackage[T1]{fontenc}
\usepackage[colorlinks=false,urlcolor=magenta,citecolor=blue,linkcolor=blue]{hyperref}

\begin{document}

\name{{\huge\bf Steven Boada, Ph.D}\vspace*{.1in}}

\begin{resume}
\vspace*{-2mm}
\section{Contact Information}
\begin{tabular}{@{}p{4.45in}p{4in}}
\href{mailto:stevenboada@gmail.com}{stevenboada@gmail.com} & \href{https://linkedin.com/in/theboada}{linkedin.com/in/theboada} \\
(615) 200-0119 & \href{https://github.com/boada}{github.com/boada} \\
\end{tabular}
\vspace*{-4mm}

% \section{Profile}
% Collaborative, scientific thinker passionate about discovering and communicating nuanced insight from complicated data. Strong programming and analytical background working with large, heterogeneous, and often noisy datasets.
% \vspace*{-2.5mm}

\section{Skills}
\textbf{Machine Learning:} Linear Models, Decision Trees, SVM, Clustering, Deep Learning, Feature Engineering\\
\textbf{Statistical Methods:} Hypothesis testing, error analysis, Monte Carlo methods, maximum likelihood\\
\textbf{Software and Computing:}  Python (e.g. Scikit-learn, Numpy, Scipy, Pandas, Matplotlib, PyTorch), mySQL, ANSI C, Linux Command Line Environments, Microsoft Office Suite, GPGPU, and HPC applications\\
\textbf{Leadership:} Demonstrable ability to tackle loosely defined problems, 5+ years organizing workflows from group planning sessions through implementation and delivery of final products, Eagle Scout. \\
\vspace*{-7mm}

\section{Professional \newline Experience}
\textbf{Insight Data Science}, New York, New York USA \newline
\textit{Fellow} \hfill \textbf{January, 2020 - Present}\newline
    \begin{list2}
    	\vspace*{-5mm}
    	\item Addressed a shortage of NYC health inspectors which caused critical health violations to remain unaddressed for extended periods of time potentially harming the general public.
    	\item Trained a transparent model to prioritize NYC restaurants based on environmental variables and their past inspection histories.
    	\item Resulted in a XX\% improved performance of NYC inspectors, leading to critical violations being discovered XX days earlier than previously expected..
    \end{list2}
\vspace*{-3mm}

\textbf{Dept. of Physics and Astronomy, Rutgers University}, New Brunswick, New Jersey USA \newline
\textit{Postdoctoral Research Associate} \hfill \textbf{September, 2016 - 2020}\newline
    \begin{list2}
    	\vspace*{-5mm}
    	\item Designed and built parallelized pipelines to process and analyze TBs of astronomical imaging; producing calibrated, standardized data catalogs and rigorous results leading to 2 peer reviewed publications and several hundred hours of telescope time.
    	\item Project managed and coordinated a team of 4, including both senior scientists and graduate students,
      to perform quality control tasks; deliver science products; and produce peer-reviewed publications.
    	\item Contributed to widely used (in astronomy), open source, Python projects through bug fixes and feature additions: see \href{https://github.com/boada/photometrypipeline}{\sc{photometrypipeline}}, \href{http://astlib.sourceforge.net/}{\sc{astLib}}, and \href{https://github.com/boada/easyGalaxy}{\sc{easyGalaxy}} on Github as examples.
    \end{list2}
\vspace*{-3mm}

\textbf{Texas A\&M University}, College Station, Texas USA\newline
\textit{Ph.D Candidate} \hfill \textbf{August, 2010 - 2016}\newline
    \begin{list2}
    	\vspace*{-5mm}
      \item Demonstrated that traditional statistical methods could be improved by up to a factor of 3, when combined with machine learning, specifically for a (then planned) large observation campaign.
      \item Implemented these machine learning methods and produced improved results in a pilot survey of the real sky and under real-world conditions.
    	\item Collaborated with group members both in person, and through collaborative tools (e.g., GitHub, SVN).
    	\item Presented scientific results in high-impact, peer reviewed journals and at international conferences.
    \end{list2}
\vspace*{-3mm}

% \textbf{The University of Tennessee}, Knoxville, Tennessee USA\newline
% \textit{Master's Candidate} \hfill \textbf{August, 2007 - 2009}\newline
%     \begin{list2}
%     	\vspace*{-5mm}
%       \item Implemented a C-based pipeline to process hundreds of GBs of simulation results. Including a computer vision algorithm to automatically analyze and compare results to expected targets.
%       \item Optimized simulation parameters using a genetic algorithm based search utilizing HPC (100k+ core) systems at the National Center for Computational Science, part of Oak Ridge National Laboratory
%     \end{list2}
% \vspace*{-3mm}

\section{Data Projects}
\textbf{Using Imaging to Infer Galaxy Properties}\newline
    \begin{list2}
    	\vspace*{-5mm}
      \item Predicted galaxy chemical composition with $\sim$5\% error from pseudo-three color imaging, a result better than other current, similar efforts in the literature.
    	\item Leveraged Convolution Neural Networks, trained on GPUs, to analyze $\sim$150,000 images from the Sloan Digital Sky Survey.
      \item Project start to publication: 4 months. See: \href{https://github.com/boada/galaxy-cnns}{github.com/boada/galaxy-cnns}
    \end{list2}
    \vspace*{-4mm}
\textbf{Predicting Tournament Performance in Warmachine}\newline
    \begin{list2}
    	\vspace*{-5mm}
    	\item Created an \href{https://en.wikipedia.org/wiki/Elo_rating_system}{Elo} based model to forecast the results of upcoming tournaments and to identify potential future upsets.
    	\item Wrangled $\sim$1800 tournament game results of the popular tabletop game using Python (e.g., Pandas).
    \end{list2}
\vspace*{-2mm}


%section for two column education
\section{Education}
\begin{tabular}{@{}p{3in}p{3in}}
  \textbf{Texas A\&M University}, College Station, Texas
  \begin{list2}
  	\item Ph.D, Physics (astronomy focus), August, 2016
  \end{list2} &
  \textbf{The University of Tennessee}, Knoxville, Tennessee
  \begin{list2}
  	\item M.S., Physics (astronomy focus),  August, 2009
  	\item B.S., Physics,  May, 2007
  \end{list2} \\
\end{tabular}
\vspace*{-4mm}

\end{resume}
\end{document}
