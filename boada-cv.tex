\documentclass[margin,line, 11pt]{res}

\topmargin=-0.5in    % Make letterhead start about 1 inch from top of page
\textheight=10in  % text height can be bigger for a longer letter
\oddsidemargin -.5in
%\evensidemargin -.5in
\textwidth=6.0in
\itemsep=0in
\parsep=0in

\nofiles

% if using pdflatex:
\setlength{\pdfpagewidth}{\paperwidth}
\setlength{\pdfpageheight}{\paperheight}

\newenvironment{list1}{
  \begin{list}{\ding{113}}{%
      \setlength{\itemsep}{0in}
      \setlength{\parsep}{0in} \setlength{\parskip}{0in}
      \setlength{\topsep}{0in} \setlength{\partopsep}{0in}
      \setlength{\leftmargin}{0.17in}}}{\end{list}}
\newenvironment{list2}{
  \begin{list}{$\bullet$}{%
      \setlength{\itemsep}{0in}
      \setlength{\parsep}{0in} \setlength{\parskip}{0in}
      \setlength{\topsep}{0in} \setlength{\partopsep}{0in}
      \setlength{\leftmargin}{0.2in}}}{\end{list}}

\renewcommand{\familydefault}{\sfdefault}
%\usepackage[sfdefault, light, condensed]{roboto}
\usepackage[light,condensed]{iwona}
\usepackage[T1]{fontenc}
\usepackage[colorlinks=false,urlcolor=magenta,citecolor=blue,linkcolor=blue]{hyperref}

\begin{document}

\name{{\huge\bf Steven Boada, Ph.D}\vspace*{.1in}}

\begin{resume}
%\vspace*{-3mm}
\section{Contact Information}
\begin{tabular}{@{}p{3in}p{3in}}
Cranford, New Jersey, USA & \href{mailto:stevenboada@gmail.com}{stevenboada@gmail.com} \\
(615) 200-0119   & \url{http://boada.github.io} \\
\end{tabular}
\vspace*{-3mm}

\section{Profile}
Collaborative, scientific thinker passionate about discovering and communicating nuanced insight from complicated data. Strong programming and analytical background working with large, heterogeneous, and often noisy datasets.
\vspace*{-2.5mm}

\section{Education}
\textbf{Texas A\&M University}, College Station, Texas USA\\
\vspace*{-4mm}
\begin{list2}
	\item Ph.D, Physics (astronomy focus), August, 2016
\end{list2}
\vspace*{-4mm}

\textbf{The University of Tennessee}, Knoxville, Tennessee USA\\
\vspace*{-4mm}
\begin{list2}
	\item M.S., Physics (astronomy focus),  August, 2009
	\item B.S., Physics,  May, 2007
\end{list2}
\vspace*{-2.5mm}

\section{Technical Skills}
\textbf{Machine Learning:} Regression (linear, logistic), Random Forests, SVM, Clustering, Feature Engineering, Optimization, Deep Learning\\
\textbf{Statistical Methods:} Hypothesis testing and confidence intervals, error analysis, image analysis, Monte Carlo methods (e.g., \href{https://github.com/dfm/emcee}{emcee}), maxiumum likelihood\\
\textbf{Software and Computing:}  Python (e.g. Scikit-learn, Numpy, Scipy, Pandas, Matplotlib, fast.ai), mySQL, ANSI C, Linux, Microsoft Excel, GPGPU, and HPC (100k+ core) applications\\
\vspace*{-8mm}

\section{Data Projects}
\textbf{Using Imaging to Predict Galaxy Properties}\newline
    \begin{list2}
    	\vspace*{-5mm}
      \item Predicted galaxy chemical composition with $\sim5\%$ error from pseudo-three color imaging.
    	\item Leveraged Convolution Neural Networks, trained on GPUs, to analyze $\sim150,000$ images from the Sloan Digital Sky Survey. See \url{https://github.com/boada/galaxy-cnns}.
    \end{list2}
    \vspace*{-4mm}
\textbf{Predicting Tournament Performance in Warmachine}\newline
    \begin{list2}
    	\vspace*{-5mm}
    	\item Wrangled tournament results of a popular tabletop game using Python (e.g., Pandas).
    	\item Created an Elo based model to forecast the results of an upcoming tournament.
    	\item Explored tournament entries for insights into and to identify potential problematic future matches.
    \end{list2}
\vspace*{-2mm}

\section{Professional \newline Experience}
\textbf{Rutgers University}, New Brunswick, New Jersey USA \newline
\textit{Postdoctoral Research Associate} \hfill \textbf{September, 2016 - Present}\newline
    \begin{list2}
    	\vspace*{-5mm}
    	\item Designed and built a Python pipeline to analyze TBs of astronomical imaging; producing calibrated, standardized data catalogs.
    	\item Led weekly collaboration meetings with senior scientists and supervised four students' research.
    	\item Contributed to open source Python projects including: \href{https://github.com/boada/photometrypipeline}{\sc{photometrypipeline}}, \href{http://astlib.sourceforge.net/}{\sc{astLib}}, and \href{https://github.com/boada/easyGalaxy}{\sc{easyGalaxy}}.
    \end{list2}
\vspace*{-3mm}

\textbf{Texas A\&M University}, College Station, Texas USA\newline
\textit{Ph.D Candidate} \hfill \textbf{August, 2010 - 2016}\newline
    \begin{list2}
    	\vspace*{-5mm}
      \item Proved simulated results for an upcoming astronomical survey could be improved, by a factor of $\sim3$, over in-house statistical methods by using Random Forest regression. Implemented these ML methods and produced improved results in a pilot survey of the real sky and under real-world conditions.
    	\item Collaborated with group members both in person, and through collaborative tools (e.g., GitHub, SVN).
    	\item Presented scientific results in high-impact, peer reviewed journals and at international conferences.
    \end{list2}
\vspace*{-3mm}

\textbf{The University of Tennessee}, Knoxville, Tennessee USA\newline
\textit{Master's Candidate} \hfill \textbf{August, 2007 - 2009}\newline
    \begin{list2}
    	\vspace*{-5mm}
      \item Implemented a computer vision algorithm to automatically analyze hundreds of GBs of simulation results.
      \item Optimized scientific simulation initial conditions using a genetic algorithm based search utilizing HPC ($\sim100$k cores) systems at the National Center for Computational Science, part of Oak Ridge National Laboratory
    \end{list2}
\vspace*{-3mm}

\end{resume}
\end{document}
