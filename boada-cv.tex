\documentclass[margin,line, 11pt]{res}

\topmargin=-0.5in    % Make letterhead start about 1 inch from top of page
\textheight=10in  % text height can be bigger for a longer letter
\oddsidemargin -.5in
%\evensidemargin -.5in
\textwidth=6.0in
\itemsep=0in
\parsep=0in

\nofiles

% if using pdflatex:
\setlength{\pdfpagewidth}{\paperwidth}
\setlength{\pdfpageheight}{\paperheight}

\newenvironment{list1}{
  \begin{list}{\ding{113}}{%
      \setlength{\itemsep}{0in}
      \setlength{\parsep}{0in} \setlength{\parskip}{0in}
      \setlength{\topsep}{0in} \setlength{\partopsep}{0in}
      \setlength{\leftmargin}{0.17in}}}{\end{list}}
\newenvironment{list2}{
  \begin{list}{$\bullet$}{%
      \setlength{\itemsep}{0in}
      \setlength{\parsep}{0in} \setlength{\parskip}{0in}
      \setlength{\topsep}{0in} \setlength{\partopsep}{0in}
      \setlength{\leftmargin}{0.2in}}}{\end{list}}

\renewcommand{\familydefault}{\sfdefault}
%\usepackage[sfdefault, light, condensed]{roboto}
\usepackage[light,condensed]{iwona}
\usepackage[T1]{fontenc}
\usepackage[colorlinks=false,urlcolor=magenta,citecolor=blue,linkcolor=blue]{hyperref}

\begin{document}

\name{{\huge\bf Steven Boada, Ph.D}\vspace*{.1in}}

\begin{resume}
\vspace*{-2mm}
\section{Contact Information}
\begin{tabular}{@{}p{3in}p{3in}}
Cranford, New Jersey, USA & \href{mailto:stevenboada@gmail.com}{stevenboada@gmail.com} \\
(615) 200-0119   & \url{http://boada.github.io} \\
\end{tabular}
\vspace*{-4mm}

\section{Profile}
Collaborative, scientific thinker passionate about discovering and communicating nuanced insight from complicated data. Strong programming and analytical background working with large, heterogeneous, and often noisy datasets.
\vspace*{-2.5mm}

\section{Technical Skills}
\textbf{Machine Learning:} Regression (linear, logistic), Random Forests, SVM, Clustering, Feature Engineering, Optimization, Deep Learning\\
\textbf{Statistical Methods:} Hypothesis testing and confidence intervals, error analysis, image analysis, Monte Carlo methods (e.g., \href{https://github.com/dfm/emcee}{emcee}), maxiumum likelihood\\
\textbf{Software and Computing:}  Python (e.g. Scikit-learn, Numpy, Scipy, Pandas, Matplotlib, fast.ai), mySQL, ANSI C, Linux Command Line Environments, Microsoft Excel, GPGPU, and HPC (100k+ core) applications\\
\vspace*{-8mm}

\section{Data Projects}
\textbf{Using Imaging to Infer Galaxy Properties}\newline
    \begin{list2}
    	\vspace*{-5mm}
      \item Predicted galaxy chemical composition with $\sim5\%$ error from pseudo-three color imaging, a result better than other current, similar efforts in the literature.
    	\item Leveraged Convolution Neural Networks, trained on GPUs, to analyze $\sim150,000$ images from the Sloan Digital Sky Survey. See \url{https://github.com/boada/galaxy-cnns}.
    \end{list2}
    \vspace*{-4mm}
\textbf{Predicting Tournament Performance in Warmachine}\newline
    \begin{list2}
    	\vspace*{-5mm}
    	\item Created an \href{https://en.wikipedia.org/wiki/Elo_rating_system}{Elo} based model to forecast the results of an upcoming tournament and to identify potential future upsets.
    	\item Wrangled $\sim1800$ tournament game results of a popular tabletop game using Python (e.g., Pandas).
    \end{list2}
\vspace*{-2mm}

\section{Professional \newline Experience}
\textbf{Dept. of Physics and Astronomy, Rutgers University}, New Brunswick, New Jersey USA \newline
\textit{Postdoctoral Research Associate} \hfill \textbf{September, 2016 - Present}\newline
    \begin{list2}
    	\vspace*{-5mm}
    	\item Designed and built massive, parallelized, Python pipelines to process and analyze TBs of astronomical imaging; producing calibrated, standardized data catalogs and rigorous results.
    	\item Coordinated a team of 4, including both senior scientists and graduate students, to perform quality control tasks; deliver science products; and produce peer-reviewed publications.
    	\item Contributed to open source Python projects including: \href{https://github.com/boada/photometrypipeline}{\sc{photometrypipeline}}, \href{http://astlib.sourceforge.net/}{\sc{astLib}}, and \href{https://github.com/boada/easyGalaxy}{\sc{easyGalaxy}}.
    \end{list2}
\vspace*{-3mm}

\textbf{Texas A\&M University}, College Station, Texas USA\newline
\textit{Ph.D Candidate} \hfill \textbf{August, 2010 - 2016}\newline
    \begin{list2}
    	\vspace*{-5mm}
      \item Proved simulated results for an upcoming astronomical survey could be improved, by a factor of $\sim3$, over in-house statistical methods by using Random Forest regression. Implemented these ML methods and produced improved results in a pilot survey of the real sky and under real-world conditions.
    	\item Collaborated with group members both in person, and through collaborative tools (e.g., GitHub, SVN).
    	\item Presented scientific results in high-impact, peer reviewed journals and at international conferences.
    \end{list2}
\vspace*{-3mm}

\textbf{The University of Tennessee}, Knoxville, Tennessee USA\newline
\textit{Master's Candidate} \hfill \textbf{August, 2007 - 2009}\newline
    \begin{list2}
    	\vspace*{-5mm}
      \item Implemented a C-based pipeline to process hundreds of GBs of simulation results. Including a computer vision algorithm to automatically analyze and compare results to expected targets.
      \item Optimized simulation parameters using a genetic algorithm based search utilizing HPC (100k+ core) systems at the National Center for Computational Science, part of Oak Ridge National Laboratory
    \end{list2}
\vspace*{-3mm}

\section{Education}
\textbf{Texas A\&M University}, College Station, Texas USA\\
\vspace*{-4mm}
\begin{list2}
	\item Ph.D, Physics (astronomy focus), August, 2016
\end{list2}
\vspace*{-4mm}

\textbf{The University of Tennessee}, Knoxville, Tennessee USA\\
\vspace*{-4mm}
\begin{list2}
	\item M.S., Physics (astronomy focus),  August, 2009
	\item B.S., Physics,  May, 2007
\end{list2}
\vspace*{-2.5mm}

\end{resume}
\end{document}

% section for two column education
% \section{Education}
% \begin{tabular}{@{}p{3in}p{3in}}
%   \textbf{Texas A\&M University}, College Station, Texas
%   \begin{list2}
%   	\item Ph.D, Physics (astronomy focus), August, 2016
%   \end{list2} &
%   \textbf{The University of Tennessee}, Knoxville, Tennessee
%   \begin{list2}
%   	\item M.S., Physics (astronomy focus),  August, 2009
%   	\item B.S., Physics,  May, 2007
%   \end{list2} \\
% \end{tabular}
% \vspace*{-4mm}
