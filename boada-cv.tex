\documentclass[margin,line, 11pt]{res}

\topmargin=-0.5in    % Make letterhead start about 1 inch from top of page
\textheight=10in  % text height can be bigger for a longer letter
\oddsidemargin -.5in
%\evensidemargin -.5in
\textwidth=6.0in
\itemsep=0in
\parsep=0in

\nofiles

% if using pdflatex:
\setlength{\pdfpagewidth}{\paperwidth}
\setlength{\pdfpageheight}{\paperheight}

\newenvironment{list1}{
  \begin{list}{\ding{113}}{%
      \setlength{\itemsep}{0in}
      \setlength{\parsep}{0in} \setlength{\parskip}{0in}
      \setlength{\topsep}{0in} \setlength{\partopsep}{0in}
      \setlength{\leftmargin}{0.17in}}}{\end{list}}
\newenvironment{list2}{
  \begin{list}{$\bullet$}{%
      \setlength{\itemsep}{0in}
      \setlength{\parsep}{0in} \setlength{\parskip}{0in}
      \setlength{\topsep}{0in} \setlength{\partopsep}{0in}
      \setlength{\leftmargin}{0.2in}}}{\end{list}}

\renewcommand{\familydefault}{\sfdefault}
%\usepackage[sfdefault, light, condensed]{roboto}
\usepackage[light,condensed]{iwona}
\usepackage[T1]{fontenc}
\usepackage[colorlinks=false,urlcolor=magenta,citecolor=blue,linkcolor=blue]{hyperref}

\begin{document}

\name{{\huge\bf Steven Boada, Ph.D}\vspace*{.1in}}

\begin{resume}
\section{\sc Contact Information}
\vspace{.05in}
\begin{tabular}{@{}p{3in}p{3in}}
Department of Physics and Astronomy & {\it Phone:}  +1 (615) 200-0119 \\
136 Frelinghuysen Rd   & {\it E-mail:}  boada@physics.rutgers.edu \\
Rutgers University & {\it WWW:} http://boada.github.io \\
Piscataway, NJ 08854  & \\
\end{tabular}

\section{\sc Research Interests}
Observation Cosmology, Large-area Sky Surveys (e.g., DES, LSST, SDSS, ACT, SPT), Galaxy Clusters, High Performance Computing (HPC), Galaxy Evolution, Interacting Galaxies and Morphology.
\vspace*{-3mm}

\section{Education}
\textbf{Texas A\&M University}, College Station, Texas USA\\
\vspace*{-4mm}
\begin{list2}
	\item Ph.D, Physics (astronomy focus), August, 2016
\end{list2}
\vspace*{-4mm}

\textbf{The University of Tennessee}, Knoxville, Tennessee USA\\
\vspace*{-4mm}
\begin{list2}
	\item M.S., Physics (astronomy focus),  August, 2009
	\item B.S., Physics,  May, 2007
\end{list2}
\vspace*{-2.5mm}

\section{Professional \newline Experience}
\textbf{Dept. of Physics and Astronomy, Rutgers University}, New Brunswick, New Jersey USA \newline
\textit{Postdoctoral Research Associate} \hfill \textbf{September, 2016 - Present}\newline
    \begin{list2}
    	\vspace*{-5mm}
    	\item Designed and built massive, parallelized, Python pipelines to process and analyze TBs of astronomical imaging; producing calibrated, standardized data catalogs and rigorous results.
    	\item Coordinated a team of 4, including both senior scientists and graduate students, to perform quality control tasks; deliver science products; and produce peer-reviewed publications.
    	\item Contributed to open source Python projects including: \href{https://github.com/boada/photometrypipeline}{\sc{photometrypipeline}}, \href{http://astlib.sourceforge.net/}{\sc{astLib}}, and \href{https://github.com/boada/easyGalaxy}{\sc{easyGalaxy}}.
    \end{list2}
\vspace*{-3mm}

\textbf{Texas A\&M University}, College Station, Texas USA\newline
\textit{Ph.D Candidate} \hfill \textbf{August, 2010 - 2016}\newline
    \begin{list2}
    	\vspace*{-5mm}
      \item Proved simulated results for an upcoming astronomical survey could be improved, by a factor of $\sim3$, over in-house statistical methods by using Random Forest regression. Implemented these ML methods and produced improved results in a pilot survey of the real sky and under real-world conditions.
    	\item Collaborated with group members both in person, and through collaborative tools (e.g., GitHub, SVN).
    	\item Presented scientific results in high-impact, peer reviewed journals and at international conferences.
    \end{list2}
\vspace*{-3mm}

\textbf{The University of Tennessee}, Knoxville, Tennessee USA\newline
\textit{Master's Candidate} \hfill \textbf{August, 2007 - 2009}\newline
    \begin{list2}
    	\vspace*{-5mm}
      \item Implemented a C-based pipeline to process hundreds of GBs of simulation results. Including a computer vision algorithm to automatically analyze and compare results to expected targets.
      \item Optimized simulation parameters using a genetic algorithm based search utilizing HPC (100k+ core) systems at the National Center for Computational Science, part of Oak Ridge National Laboratory
    \end{list2}
\vspace*{-3mm}

\section{\sc Observing Experience}
Proposals
    \begin{list2}
        \vspace*{.05in}
    \item \emph{On the Trail of the Most Massive Galaxy Clusters in the Universe}\\
    Co-I (PI: J. Hughes), KPNO, 3 nights awarded, 2016
    \item \emph{X-ray Confirmation of Candidate Planck Clusters with Swift}\\
    Co-I (PI: J. Hughes), Swift X-ray Observatory, 2016
    \item \emph{Measuring the Masses of X-ray-Selected, Low-Mass Galaxy Clusters and Groups with Integral Field Spectroscopy}\\
		Co-I (PI: N. Mehrtens), McDonald Observatory, 4 nights awarded, 2013
    \item \emph{Measuring the Masses of Galaxy Clusters with Integral Field Spectroscopy}\\
		Co-I (PI: C. Papovich), McDonald Observatory, 9 nights awarded, 2012
    \item \emph{Measuring the Masses of Galaxy Clusters with Integral Field Spectroscopy}\\
		Co-I (PI: C. Papovich), McDonald Observatory, 5 nights awarded, 2012
	\end{list2}
Telescopes
    \begin{list2}
        \vspace*{.05in}
    \item Harlan J. Smith 2.7m Telescope, Mitchell Spectrograph (formerly VIRUS-P), 20+ nights
    \item KPNO, Mayall 4m Telescope, MOSAIC3, NEWFIRM, 10+ nights
	\end{list2}
Data Experience
    \begin{list2}
        \vspace*{.05in}
    \item Optical and Near-IR Imagine
		\item Integral Field Spectroscopy
    \item Hubble Space Telescope Imaging
		\item Sloan Digital Sky Survey Imaging and Spectroscopy
	\end{list2}

\section{\sc Computing Experience}
Extensive experience in the processing and application of large astronomical data sets, including: the acquisition and reduction of optical integral field unit spectroscopy, querying large astronomical databases such as the Sloan Digital Sky Survey and the Millennium Simulation, analysis of multi-wavelength imaging from the Hubble Space Telescope. Key computing skills include: mastery of the Python language, and the interface with other languages and tools, considerable experience with large multiprocessor applications (e.g. Gadget-2) and high performance computing systems, supervised and unsupervised machine learning and optimization, GPGPU computing, and participation in open source and collaborative development environments, including version control. Contributor to {\sc AstroPy}. Co-author of \href{http://astlib.sourceforge.net/}{\sc{astLib}} Python library.
\vspace*{-3mm}

\section{\sc Teaching and Outreach}
\textbf{Texas A\&M University}, College Station, Texas USA
\vspace{-3mm}

\emph{Teaching Assistant}\hfill \textbf{2010 - Spring, 2015}\\
Supervised undergraduate students for weekly lab sessions, tutoring sessions, grading of homework and quizzes for Basic Astronomy, Overview of Modern Astronomy, and Survey of Astronomy.
\vspace*{-3mm}

\emph{Physics Festival} \hfill \textbf{2010 - Present}\\
Demonstrated physics and astronomy principles for students from elementary through high school and the general public.
\vspace*{-3mm}

\emph{Star Parties} \hfill \textbf{2010 - Present}\\
Discussed astronomical topics and operated telescopes for college students and the general public.
%\vspace*{3mm}

%\pagebreak
\textbf{Nashville State Community College}, Nashville, Tennessee USA
\vspace{-3mm}

\emph{Adjunct Faculty} \hfill \textbf{Spring, 2010}\\
Primary instructor for introductory physics course, Conceptual Physics.
%\vspace*{3mm}

\textbf{The University of Tennessee}, Knoxville, Tennessee USA
\vspace{-3mm}

\emph{Teaching Assistant} \hfill \textbf{August, 2007 - 2009}\\
Supervised laboratory experiences for undergraduate students in Introduction to Modern Physics, and Electricity and Magnetism for Engineering. Designed and taught laboratories for undergraduate Honors Astronomy.
\vspace*{-3mm}

\section{\sc Academic Honors \\and Awards}
The University of Tennessee: graduated Magna Cum Laude, Phi Beta Kappa, Sigma
Pi Sigma, President, Society of Physics Students 2006 thru 2007
%\vspace*{-1mm}

\section{\sc Grants and Awards}
\begin{list2}
    \item \emph{The Road to the Virgo Cluster: The DECam/IRAC Galaxy Environment Survey}\\
	Co-I (PI: C. Papovich), NSF Alliances for Graduate Education and the Professoriate, 2015
	\item \emph{Graduate Student Presentation Grant}\\
	PI, Texas A\&M University Office of Graduate and Professional Studies, 2015
	\item \emph{Graduate Student Travel Grant}\\
	PI, Texas A\&M University Department of Physics and Astronomy, 2015
\end{list2}

\section{\sc Posters and Presentations}
Talk: Galaxy Cluster Workshop, Center for Computational Astrophysics, NYC June 2018\\
Talk: Tri-State Postdoc Retreat, Center for Computational Astrophysics, NYC May 2018\\
Talk: Tri-State Postdoc Retreat, Columbia University, NYC March 2017\\
Talk: Astronomy Seminar Series, Rutgers University, New Brunswick, NJ October 2016\\
Talk: 227th AAS Meeting, Kissimmee, FL January, 2016\\
Talk: CANDELS Team Meeting, University of Santa Cruz, Santa Cruz, CA July, 2015\\
Talk: CANDELS Team Meeting, STScI, Baltimore, MD July, 2014\\
Poster: Bashfest Symposium, University of Texas, Austin, TX October, 2013\\
Talk: CANDELS Team Meeting, University of Kentucky, Lexington, KY August, 2013\\
Poster: GMT Science Meeting, University of Chicago, Chicago, IL June, 2013\\
Talk: CANDELS Team Meeting, University of Santa Cruz, Santa Cruz, CA September, 2012\\
Poster: 219th AAS Meeting, Austin, TX January, 2012\\
Poster: Bashfest Symposium, University of Texas, Austin, TX October, 2011\\
Talk: Texas A\&M Astronomy Symposium, Texas A\&M University, College Station, TX August, 2011--15\\

\section{\sc References}
Available upon request.

\section{\sc Publications}
\begin{list2}
\item Li, T.~S., DePoy, D.~L., Marshall, J.~L., Tucker, D., Kessler, R., Annis, J.,
Bernstein, G.~M., \textbf{Boada, S.}, Burke, D.~L., Finley, D.~A., James, D.~J., Kent, S.,
Lin, H., Marriner, J., Mondrik, N., Nagasawa, D., Rykoff, E.~S., Scolnic, D., Walker, A.~R.,
Wester, W., Abbott, T.~M.~C., Allam, S., Benoit-L\'evy, A., Bertin, E., Brooks, D., Capozzi,
D., Carnero Rosell, A., Carrasco Kind, M., Carretero, J., Crocce, M., Cunha, C.~E.,
D'Andrea, C.~B., da Costa, L.~N., Desai, S., Diehl, H.~T., Doel, P., Flaugher, B., Fosalba,
P., Frieman, J., Gaztanaga, E., Goldstein, D.~A., Gruen, D., Gruendl, R.~A., Gutierrez, G.,
Honscheid, K., Kuehn, K., Kuropatkin, N., Maia, M.~A.~G., Melchior, P., Miller, C.~J.,
Miquel, R., Mohr, J.~J., Neilsen, E., Nichol, R.~C., Nord, B., Ogando, R., Plazas, A.~A.,
Romer, A.~K., Roodman, A., Sako, M., Sanchez, E., Scarpine, V., Schubnell, M.,
Sevilla-Noarbe, I., Smith, R.~C., Soares-Santos, M., Sobreira, F., Suchyta, E., Tarle, G.,
Thomas, D., Vikram, V., and The DES Collaboration \emph{Assessment of Systematic Chromatic
Errors that Impact Sub-1\% Photometric Precision in Large-Area Sky Surveys} \href{http://adsabs.harvard.edu/abs/2016AJ....151..157L}{2016, ApJ, 151, 157}

\item \textbf{Steven Boada}, Tilvi, V., Papovich, C., Quadri, R. F., Hilton, M., Finkelstein, S., Guo, Y., Bond, N., Conselice, C., Dekel, A., Ferguson, H., Giavalisco, M., Grogin, N. A., Kocevski, D. D., Koekemoer, A. M. and Koo, D. C. \emph{The Role of Bulge Formation in the Homogenization of Stellar Populations at $z\sim2$ as revealed by Internal Color Dispersion in CANDELS} \href{http://adsabs.harvard.edu/abs/2015ApJ...803..104B}{2015, ApJ, 803, 104}
\end{list2}

\section{\sc Conference Proceedings}
\begin{list2}
\item Ting Li, DePoy, D. L., Marshall, Jennifer L., Nagasawa, D. Q., Carona, D. W., {\bf Boada, S.} \emph{Monitoring the atmospheric throughput at Cerro Tololo Inter-American Observatory with aTmCam} 2014, Proceedings of the SPIE, 9147, 91476Z
\end{list2}


\end{resume}
\end{document}
