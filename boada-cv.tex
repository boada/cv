\documentclass[margin,line, 11pt]{res}

\topmargin=-0.5in    % Make letterhead start about 1 inch from top of page
\textheight=10in  % text height can be bigger for a longer letter
\oddsidemargin -.5in
\evensidemargin -.5in
\textwidth=6.5in
\itemsep=0in
\parsep=0in

\nofiles

% if using pdflatex:
\setlength{\pdfpagewidth}{\paperwidth}
\setlength{\pdfpageheight}{\paperheight}

\newenvironment{list1}{
  \begin{list}{\ding{113}}{%
      \setlength{\itemsep}{0in}
      \setlength{\parsep}{0in} \setlength{\parskip}{0in}
      \setlength{\topsep}{0in} \setlength{\partopsep}{0in}
      \setlength{\leftmargin}{0.17in}}}{\end{list}}
\newenvironment{list2}{
  \begin{list}{$\bullet$}{%
      \setlength{\itemsep}{0in}
      \setlength{\parsep}{0in} \setlength{\parskip}{0in}
      \setlength{\topsep}{0in} \setlength{\partopsep}{0in}
      \setlength{\leftmargin}{0.2in}}}{\end{list}}

\renewcommand{\familydefault}{\sfdefault}
%\usepackage[sfdefault, light, condensed]{roboto}
\usepackage[light,condensed]{iwona}
\usepackage[T1]{fontenc}
\usepackage[colorlinks=false,urlcolor=magenta,citecolor=blue,linkcolor=blue]{hyperref}

\begin{document}

\name{{\huge\bf Steven Boada, Ph.D}\vspace*{.1in}}

\begin{resume}
\vspace*{-2mm}
\section{Basic\\Information}
\begin{tabular}{@{}p{4.9in}p{4in}}
  (615) 200-0119 & \href{https://github.com/boada}{github.com/boada} \\
  \href{mailto:stevenboada@gmail.com}{stevenboada@gmail.com} & \href{https://linkedin.com/in/theboada}{linkedin.com/in/theboada} \\
\end{tabular}
\vspace*{-4mm}

% \section{Profile}
% Collaborative, scientific thinker passionate about discovering and communicating nuanced insight from complicated data. Strong programming and analytical background working with large, heterogeneous, and often noisy datasets.
% \vspace*{-2.5mm}

\section{Skills}
\textbf{Machine Learning:} Linear Models, Decision Trees, SVM, Clustering, Deep Learning, Feature Engineering\\
\textbf{Statistical Methods:} Hypothesis testing, error analysis, Monte Carlo methods, maximum likelihood\\
\textbf{Software and Computing:}  Python (e.g. Scikit-learn, Numpy, Scipy, Pandas, Matplotlib, PyTorch), mySQL, ANSI C, Linux Command Line Environments, GPGPU, and HPC applications, AWS \\
\textbf{Leadership:} Experience organizing and leading workshops and collaboration meetings, Teaching and mentoring junior team members, Eagle Scout. \\
\vspace*{-7mm}

\section{Professional \newline Experience}
\textbf{Insight Data Science}, New York, New York USA \newline
\textit{Fellow} \hfill \textbf{January, 2020 - Present}\newline
    \begin{list2}
    	\vspace*{-5mm}
      \item Helped optimize the way NYC health inspectors perform restaurant inspections in order to reduce the time critical health violations remain unaddressed.
    	\item Trained a random forest in Python to prioritize NYC restaurant inspections based on environmental variables and their past inspection histories and provided the results to NYC through an API deployed on AWS.
    	\item Resulted in NYC inspectors identifying $\sim$2.5\% more violations in the first half of an inspection window,  leading to critical violations being discovered up to 7 days earlier than by the current approach implemented by NYC.
    \end{list2}
\vspace*{-2mm}

\textbf{Dept. of Physics and Astronomy, Rutgers University}, New Brunswick, New Jersey USA \newline
\textit{Postdoctoral Research Associate} \hfill \textbf{September, 2016 - 2020}\newline
    \begin{list2}
    	\vspace*{-5mm}
    	\item Designed and built parallelized pipelines to process and analyze TBs of astronomical imaging; producing calibrated, standardized data catalogs and rigorous results leading to 2 peer reviewed publications and several hundred hours of telescope time.
    	\item Project managed and coordinated a team of 4, including both senior scientists and graduate students,
      to perform quality control tasks; deliver science products; and produce peer-reviewed publications.
    	\item Contributed to open source, astronomy-focused, Python projects through bug fixes and feature additions: see \href{https://github.com/boada/photometrypipeline}{photometrypipeline}, \href{http://astlib.sourceforge.net/}{astLib}, and \href{https://github.com/boada/easyGalaxy}{easyGalaxy} on GitHub as examples.
    \end{list2}
\vspace*{-2mm}

\textbf{Dept. of Physics and Astronomy, Texas A\&M University}, College Station, Texas USA\newline
\textit{Ph.D Candidate} \hfill \textbf{August, 2010 - 2016}\newline
    \begin{list2}
    	\vspace*{-5mm}
      \item Demonstrated that measurements from a planned large observation campaign could be improved by up to a factor of 3 over traditional statistical methods through the use of machine learning.
      \item Implemented these machine learning methods and produced reliable results in a pilot survey of the real sky and under real-world conditions.
    	\item Collaborated with group members both in person, and through collaborative tools (e.g., GitHub, SVN).
    	\item Presented scientific results in high-impact, astrophysical journals and at international conferences.
    \end{list2}
\vspace*{-2mm}

% \textbf{The University of Tennessee}, Knoxville, Tennessee USA\newline
% \textit{Master's Candidate} \hfill \textbf{August, 2007 - 2009}\newline
%     \begin{list2}
%     	\vspace*{-5mm}
%       \item Implemented a C-based pipeline to process hundreds of GBs of simulation results. Including a computer vision algorithm to automatically analyze and compare results to expected targets.
%       \item Optimized simulation parameters using a genetic algorithm based search utilizing HPC (100k+ core) systems at the National Center for Computational Science, part of Oak Ridge National Laboratory
%     \end{list2}
% \vspace*{-3mm}

\section{Data Projects}
\textbf{Using Imaging to Infer Galaxy Properties}\newline
    \begin{list2}
    	\vspace*{-5mm}
      \item Predicted galaxy chemical composition with $\sim$5\% error from pseudo-three color imaging, a result better than other current, similar efforts in the literature.
    	\item Leveraged Convolution Neural Networks, trained on GPUs, to analyze $\sim$150,000 images from the Sloan Digital Sky Survey.
      \item Project start to publication: 4 months (typically $\sim$1.5 years). See: \href{https://github.com/boada/galaxy-cnns}{github.com/boada/galaxy-cnns}.
    \end{list2}
    \vspace*{-3mm}

\textbf{Predicting Tournament Performance in Warmachine}\newline
    \begin{list2}
    	\vspace*{-5mm}
    	\item Created an \href{https://en.wikipedia.org/wiki/Elo_rating_system}{Elo} based model to forecast the results of upcoming tournaments and identify potential upsets.
    	\item Integrated predictions into a local community ranking system and forecasted $\sim$1800 tournament game results of the popular tabletop game using Python (e.g., Pandas).
    \end{list2}
\vspace*{-1mm}

%section for two column education
\section{Education}
\begin{tabular}{@{}p{3in}p{3in}}
  \textbf{Texas A\&M University}, College Station, Texas
  \begin{list2}
  	\item Ph.D, Physics (astronomy focus), 2016
  \end{list2} &
  \textbf{The University of Tennessee}, Knoxville, Tennessee
  \begin{list2}
  	\item M.S., Physics (astronomy focus), 2009
  	\item B.S., Physics, 2007
  \end{list2} \\
\end{tabular}
\vspace*{-4mm}

\end{resume}
\end{document}
